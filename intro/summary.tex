% \pagestyle{plain}
% \begin{center}
% {\LARGE Περίληψη}\\[1cm]
% \end{center}

% Τις τελευταίες δεκαετίες η ασφάλεια υπολογιστών και δικτύων έχει τραβήξει το ενδιαφέρον τόσο των ερευνητών όσο και της βιομηχανίας. Το ενδιαφέρον αυτό συνεχίζει να αυξάνεται με εκθετικό ρυθμό τα τελευταία χρόνια λόγω των συνεχώς αυξανόμενων επιθέσεων, της συνεχούς μεγέθυνσης των εταιρικών και κυβερνητικών δικτύων καθώς και την ολοένα αυξανόμενη χρήση και αξιοποίηση των υπολογιστικών συστημάτων σε κάθε πτυχή της ανθρώπινης δραστηριότητας.

% Στο πολύ ενεργό αυτό πεδίο, προκύπτουν συνέχεια νέα προβλήματα και παρουσιάζονται συνεχώς νέες ιδέες για την επίλυσή τους. Μία από τις πιο υποσχόμενες είναι η σουίτα IPsec, η οποία προστατεύει την κίνηση των δικτύων στο επίπεδο IP της στοίβας πρωτοκόλλων του Internet, TCP/IP. Η εφαρμογή του έχει ήδη ξεκινήσει τα τελευταία χρόνια, σε μικρή κλίμακα, αλλά με την μετάβαση που προβλέπεται να γίνει από το IPv4 στο IPv6, η υλοποίηση του IPsec θα είναι υποχρεωτικό να υπάρχει σε κάθε δικτυακό σύστημα με την προοπτική της ενδυνάμωσης της ασφάλειας στο Internet.

% Ακόμα, η ανάπτυξη υπολογιστικών συστημάτων ειδικών εφαρμογών έχει καταφύγει τα τελευταία χρόνια στην μορφή των ενσωματωμένων συστημάτων (embedded systems). Για την σχεδίαση και πιστοποίηση της ορθής λειτουργίας αυτών των συστημάτων είναι σύνηθες να χρησιμοποιούνται FPGA (Field Programmable Gated Array) chip  ενώ η τελική μορφή του συστήματος είναι συνήθως υλοποιημένη σε ASIC (Application Specific Integrated Circuit) διότι δίνει τα πλεονεκτήματα μεγαλύτερης ταχύτητας και μικρότερης κατανάλωσης ενέργειας σε σχέση με τα FPGA.

% Στην  παρούσα  διπλωματική  εργασία  σχεδιάστηκε το  πρωτόκολλο  IPsec  ως  ένα ενσωματωμένο σύστημα υλικού-λογισμικού και υλοποιήθηκε  στην πλατφόρμα FPGA Virtex 5 της εταιρείας Xilinx. Το ενσωματωμένο σύστημα εμπεριέχει έναν επεξεργαστή Microblaze και επιταγχυντές υλικού (hardware accelerators). Η υλοποίηση έγινε με αποδοτική συσχεδίαση υλικού και λογισμικού ώστε να γίνεται αξιοποίηση των πλεονεκτημάτων και των δύο. Συγκεκριμένα, σε υλικό σχεδιάστηκαν οι, απαιτητικοί σε χρόνο, κρυπτογραφικοί πυρήνες του  συστήματος,  CBC-AES-128  και  HMAC-SHA1-96, ενώ το υπόλοιπο τμήμα του IPsec σχεδιάστηκε σε λογισμικό. Για την σχεδίαση και υλοποίηση του ακολουθήθηκαν οι προδιαγραφές  που  δίνονται  στα  αντίστοιχα Data-Sheets και RFCs (Request  For  Comments) και έγινε προσπάθεια να υλοποιηθεί όσο το δυνατόν μεγαλύτερο μέρος αυτών και με όση το δυνατόν ακρίβεια. Τέλος, έγινε on-chip πιστοποίηση ορθής λειτουργίας του συστήματος στην αναπτυξιακή πλακέτα ML507 (Virtex-5) με σύνδεση της σε ένα δίκτυο υπολογιστών και κρυπτογράφηση/αποκρυπτογράφηση πραγματικών πακέτων δεδομένων.


% \clearemptydoublepage

\pagestyle{plain}
\begin{center}
{\LARGE Abstract}\\[1cm]
\end{center}

\noindent
Over recent decades, computer and network security have garnered substantial attention from both academia and industry, owing to the escalating frequency and magnitude of cyber threats, the expansion of corporate and governmental networks, and the pervasive reliance on computer systems across diverse domains of human activity.

In the context of numerous solutions emerging in this rapidly evolving domain, the Internet Protocol Security (IPsec) protocol suite has emerged as a highly promising approach for safeguarding network traffic at the IP layer of the TCP/IP internet protocol stack. While its adoption initially commenced on a limited scale, the impending transition from IPv4 to IPv6 mandates the integration of IPsec implementations into every networking system, signaling a significant step forward in bolstering Internet security.

Furthermore, recent trends in the development of application-specific systems have increasingly embraced embedded system solutions. Field Programmable Gate Array (FPGA) chips have gained prominence as instrumental tools in the development and validation of embedded systems, with the ultimate implementation of such systems being an Application-Specific Integrated Circuit (ASIC), due to its competitive advantages in speed and power efficiency.

This diploma thesis undertakes the task of designing an embedded system that implements the IPsec protocol suite. The hardware/software co-design methodology is applied aiming to exploit the synergy between the two to benefit from the best of both worlds. The target device is a Xilinx Virtex 5 FPGA platform. A Microblaze processor is utilized for the software execution while optimized hardware accelerators are designed for the computationally intensive cryptographic components CBC-AES-128 and HMAC-SHA1-96. The implementation is following the protocol specifications dictated by the corresponding RFCs (Request For Comments), with all mandatory features being implemented. Finally, the system is verified and evaluated within a real-world computer network environment.

